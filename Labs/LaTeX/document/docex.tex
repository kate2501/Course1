\documentclass{article}
\usepackage[utf8]{inputenc}
\usepackage[english,russian]{babel}
\usepackage{indentfirst}

\title{Образец текста}
\author{Е.\,A. Бельская}
\date{\today}

\begin{document}

\maketitle

\begin{abstract}
Это образец входного файла. Сравнивая его с готовым печатным документом, нетрудно освоить азы работы с \LaTeX'ом.
\end{abstract}

\section{Обычный текст}
Окончания слов и предложений отмечаются, как обычно, пробелами. Не имеет значения, сколько пробелов Вы наберёте; один пробел так же хорош, как и 100.

 
Одна или несколько пустых строк обозначают конец абзаца.

Поскольку любое количество пробелов рассматривается как один, способо форматирования текста во входном файле безразличен для \LaTeX а.
Однако разумное форматирование входного файла облегчает его чтение, проверку и внесение изменений.

\subsection{Математические выражения}
\LaTeX{} превосходно печатает как простые математические уравнения типа 
\(x-3y=7\), так и более сложные.
Математическую формулу можно записать отдельной строкой:
\[x'+y^{2}=z_{i}^{2}.\]
Чтобы пронумеровать формулу, используйте процедуру \texttt{equation}:
\begin{equation}
\int_{-\infty}^{\infty}dx \exp(-x^2)=\sqrt{\pi}.
\end{equation}
\begin{center}
\Large
Всё остальное Вы узнаете,\\ прочитав эту книгу.
\end{center}
\end{document}