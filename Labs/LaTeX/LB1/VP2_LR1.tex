%Преамбула
\documentclass[a4paper,12pt]{article}
\usepackage[utf8]{inputenc}
\usepackage[english,russian]{babel}
\title{Лабораторная работа №1\\Работа с текстом}
\author{БГУ,ММФ,1 курс, 5 группа, Бельская Екатерина Артуровна}
%Тело документа
\begin{document}
\maketitle
My first sentence in \LaTeX.
Тригонометрическая функция $y=\sin{x}$
Количество пробелов в тексте         не имеет значения.
Также единичным пробелом считается единичный переход на новую строку.

Пустые строки разделяют текст на абзацы.
Форматировать текст можно с помощью \underline{команд}, {\bf декларация} или \begin{em}окружений\end{em}
\begin{verbatim}
Форматировать текст можно с помощью \underline{команд}, {\bf декларация} или \begin{em}окружений\end{em}
\end{verbatim}
\newpage
\section{Глава 1}
\subsection{Входной файл}
\LaTeX{} преобразует \begin{em}исходный\end{em} текст в \begin{em}печатный\end{em} документ с помощью специальных команд, которые дают указания компилятору, как должен выглядеть печатный документ.
\subsection{Кое-что о классе документа}
Первая команда во входном файле показывает, к какому {\em классу} (их немного) будет принадлежать печатный документ.
\subsection{Пример входного документа}
Команды подключения пакетов (пакет --- это служебный текстовый файл с расширением sty) могут располагаться только в преамбуле.
\subsection{Буквы и символы}
\LaTeX{} распознает строичные и прописные буквы, цифры, 16 знаков препинания; существуют математические символы, символы для служебного пользования и невидимые символы.
\subsubsection{Знакомимся с клавиатурой}
Шрифты, которые использует \LaTeX{} для печати документа и его просмотра на экране дисплея, имеют собственную(внутреннюю) кодировку, не совпадающую с кодировками, используемых в различных операционных системах.
\subsubsection{Шрифты, литеры и лигатуры }
Современные шрифты \LaTeX'а содержат до 256 {\em литер}(изображений букв или символов)(необх. опр.:{\em лигатура} --- комбинация символов, изображаемая одной литерой;{\em кернение} --- способ печати, регулирующий ср. промежуток между соседними литерами)
\subsubsection{Специальные щрифты}
С помощью широкого набора символов можно создавать тексты нестандратного написания.
\subsection{Слова и предложения}
\LaTeX{} игнорирует то, как набран исходный текст во входном файле: он фиксирует толбко границы слов, предложений и абзацев.
\subsubsection{Кавычки}
Кавычки удобно набирать при помощи лигатур (для русских кавычек необходим пакет babel с опцией russian).
\subsubsection{Дефисы и тире}
Один символ <<->> --- дефис, два или три --- тире.
\subsubsection{Логосы}
Отдельные слова или даже предложения могут набираться отдельными символами(например:\LaTeX,\TeX,\today ).
\subsubsection{Подстрочные примечания}
Команду сноски\footnote {Это пример сноски} \verb=\footnote= необходимо вставить в том месте исходного текста, где в печатном документе должен появиться маркёр примечания, а в качестве аргумента набрать текст примечания.
\subsection{Комментарии}
Комментарий записывается с помощью символа \%; всё, что находится после \verb=end{document}= также считается комментарием.
\subsection{Строки и абзацы}
Количество пробелов в тексте не имеет значения, такжe \LaTeX{} автоматически разбивает текст на строки и перенесит слова так, чтобы пробелы между словами в соседних строках также были по возможности равны.
\subsection{Выделение текста}
Форматировать текст можно с помощью \underline{команд} (\verb=\underline{команд}=), {\bf деклараций} (\verb={\bf деклараций}=) или \begin{em}окружений\end{em} (\verb={\em окружений}=).

\subsection{Декларации}
Команды, не имеющие аргументов, а просто информирующие \LaTeX, что надо выделить текст, называются {\em декларациями}.
\subsection{Процедуры}
Пример процедуры: \verb=\begin{процедура}= тело процедуры \verb=\end{процедура}=, являются аналогами деклараций, можно создавать свои процедуры,дополнительные возможности с помощбю использования пакетов.
\subsection{Экскурсия в море шрифтов}
\LaTeX{} различает форму шрифта(начертание), серию(насыщенность) и семейство(гарнитура); выбор шрифта есть признак вызуального проектирования документа.
\subsection{Печатный документ}
Полезно-усвоить одно правило --- всегда начинать входной файл с команды  \verb=\documentclass=.
\subsubsection{Титульная страница}
Печатный документ начинается с титульной страницы, на которой располагаются его название, список авторов и, возможно, авторов. 
\subsubsection{Глава, секция и др.}
Печатный текст разбивается на лоогически завершённые части --- {\bf разделы}: главы, секции и др.(их нумерация автоматическая, они образуют иерархическую структуру).
\subsection{Диагностические сообщения}
При наличии ошибок система сообщает об этом пользователю.
\subsection{Ходит информация по кругу}
Система переработки файлов \LaTeXом очень сложная, входной файл --- это лишь вершина айсберга.

\end{document}